\section{Implementation \& Integration}
\label{sec: implementation}%
As showed in the deployment diagram, there relations between services that should be considered for the definition of
the order of implementation of the modules.
We suppose that DBMS are already deployed systems that will be used by our components.
It follows a list of a possible path for the implementation.
Each step lists modules that can be deployed in parallel.
The order is defined by the dependencies between components.
It is:
\begin{itemize}
    \item We start with the modules that directly communicate with DBMSs. They are: Registration Service,
    EVD Authentication Service, EVD Profile Manager Service, Explore Stations Service, Calendar Service, CPO Authentication Service,
    CPO Profile Manager Service, Charging Stations Manager Service.
    \item The implementation should continue with modules that allow communication with external services.
    They are: Payment Service, Charging Station Communication Service, DSO Manager Service.
    \item Now it is the turn of components that have strong relations with previous implemented service, as:
    Promotion Service, Booking Service and Charging Session Service.
    \item Finally, it has to be implemented the module that communicates with the greater number of services: Suggestion Service.
    In this final step, it should be completed the process for the asynchronous call started by Calendar Service
    after the addition of a new activity.
\end{itemize}


\section{Test Plan}
\label{sec: test_plan}%
\subsection{Functional Testing}
The system should be tested in order to actually guarantee the functionalities described in the RASD document; this can be done executing automated tests.
Performances should be checked too, and this can be done simulating real life scenarios in a real world environment, stressing the system.
\subsection{User Interface Testing}
Regarding the UI/UX, as said in the RASD document, it is crucial that the application is intuitive and as easy to use as possible, since it will be fruited by elderly and users that will be not always practical with new electronic services; to achieve this goal, the user experience will have to be tested focusing on features such as usability and accessibility. Focus groups, composed by old persons and digital non-natives, could test the journey through the app, in order to test its smoothness. Automated tests could be conducted to test scenarios described in the RASD document.