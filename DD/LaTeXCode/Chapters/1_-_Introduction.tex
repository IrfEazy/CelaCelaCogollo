\section{Purpose}
\label{sec:purpose}%
Climate change is a topical issue for everybody these days, and it's leading a change not only in our habits but also in consumes, which implies a change in the production model our society adopted until now.
Private mobility is one of the market sectors which is changing the most, directly dependent from fossil fuel.

In order to keep global warming below 1.5°C, Europe have decided to reduce greenhouse gas emissions of CO2 per
person per year by 2030, and, by the same year, the IEA predicts that electric vehicles will have a market share of roughly 30 percent, with a total number of 23 million e-cars on the roads: electric vehicles represent eco-friendly mobility solutions that are, and will be on our roads in the next future.

Most of the current electric cars can travel between 150 and 350 kilometers on a single charge, but premium-brand models can currently cover more than 500 kilometers.
This being said, it's obvious that, when people use an electric vehicle, knowing where to charge it and carefully planning the
charging process in such a way that it introduces minimal interference and constraints on our daily schedule
is of great importance.

That's were \verb|eMALL| operates: it can find charging stations owned by several Charging Point Operators - CPO - and,
considering the activities in user's schedule, it can propose the best possible path of charging process
in order to minimize the cost and the waisted time at the station.


\section{Scope}
\label{sec:scope}%
The scope of the design document is to provide a detailed description of how our system will be implemented to meet the requirements defined in the RASD\@.
It outlines the overall architecture of \verb|eMALL|, including any major components and their interactions, as well as the detailed design of each component and its interfaces with other parts of the system.
It provides a complete and thorough description of the proposed solution, including any trade-offs or assumptions that were made during the design process.


\section{Definition, Acronyms, Abbreviations}
\label{sec:definition_acronyms_abbreviations}%
\begin{table}[H]
    \begin{center}
        \begin{tabular}{ |l|l| }
            \hline
            \textbf{Acronyms} & \textbf{Definition}            \\
            \hline
            eMSP              & e-Mobility Service Provider    \\
            \hline
            CPO               & Charging Point Operator        \\
            \hline
            CPMS              & Charge Point Management System \\
            \hline
            DSO               & Distribution System Operator   \\
            \hline
            DD                & Design Document                \\
            \hline
            WPX               & World Phenomena X              \\
            \hline
            SPX               & Shared Phenomena X             \\
            \hline
            GX                & Goal Number X                  \\
            \hline
            DAX               & Domain Assumptions X           \\
            \hline
            UCX               & Use Case X                     \\
            \hline
            EVD               & Electric Vehicle Driver        \\
            \hline
            EV                & Electric Vehicle               \\
            \hline
        \end{tabular}
        \caption{Acronyms used in the document.}
        \label{tab:acronyms}%
    \end{center}
\end{table}


\section{Reference Documents}
\label{sec:reference_documents}%
\begin{itemize}
    \item \href{https://polimi365-my.sharepoint.com/:b:/g/personal/10685242_polimi_it/EWPABzzjfF9EsgYvSiuvdAIBAz6qnjdfLuPE8kwQSxeyCg?e=6qasKD}{The specification document Assignment RDD AY 2022--2023.pdf}
\end{itemize}


\section{Document Structure}
\label{sec:doc_structure}%
The document is structured in seven sections, as described below.

First section introduce the purposes;
abbreviations and definitions useful to understand the problem are listed as well.

The following section, the second one, provides the chosen architectural design for the problem: here we describe
the identified system components, their relations, the offered communication interfaces, their behavior in the system
and architectural styles and design patterns used.

Later on, the third section focuses on user interfaces, presenting and describing the mockups offered to users.

The fourth section shows how the system meets the requirements.
At first, the section provides the mapping between identified components and functional requirements listed in the third
section of RASD\@.
Then, a description of the satisfaction of performance requirements and system attributes is provided.

Lastly, the fifth section provides a plan %TODO: add a quick description of the last section.

Section six reports the effort spent by each group member in the redaction of this document, meanwhile the last
section simply lists bibliography references and other resources used to redact this document.
