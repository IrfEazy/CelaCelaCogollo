\section{External Interface Requirements}
\label{sec:external_interface_requirements}%

\subsection{User Interfaces}
\label{subsec:user_interfaces}%
The eMALL’s user interfaces are a website and a mobile application;
the first is developed to be used mainly by CPOs with a dedicated login section for businesses but can be used by EVDs too.
The mobile application is available for Android and iOS and provides an enhanced experience as compared to the website
since it offers users personalized suggestions based on their location.
The website and the app should be easy to use since they will be used mostly by middle-aged users,
who might not always be familiar with the technology.
A “quick booking” section dedicated to facilitating the booking process might be included,
for those EVDs who are used to booking a charge at the same charging station (based on suggestions given by AI).

\subsection{Hardware Interfaces}
\label{subsec:hardware_interfaces}%
The system only requires a smartphone or computer with an internet connection and web browser to access websites or mobile applications.
Furthermore, eMALL communicates with the EV through its company's API to get the current battery level, the charging state,
so if it is plugged in and if it is charging, and the number of routable kilometers obtained on the current battery level.
To access personalized suggestions, based on EV’s position, the device in use has to be able to detect its location with a GPS or Glonass localization system.

%TODO: think id there is something you can write in this subsection

\subsection{Software Interfaces}
\label{subsec:software_interfaces}%

\subsection{Communication Interfaces}
\label{subsec:communication_interfaces}%
The eMALL system needs to communicate with other actors to offer functionalities to the users;
the communication is bidirectional and permits eMALL to obtain the desired data and serve elaborated data.
Below are listed different communication interfaces used to exchange information with users:
\begin{itemize}
    \item \textbf{CPMS and Charging Points.} The CPMSs offered to the CPO will communicate with the charging point
    through the OCPP communication protocol.
    Thanks to it, the system can manage the charging session, given the possibility of starting and stopping it.
    Another significant functionality offered by OCPP is the diagnostic of the charging point:
    a CPO can reboot its charging spots, can get their log, and can update their firmware.
    %TODO: Does it make sense to put an example of an external system
    \item \textbf{eMALL and EVs.} The eMALL system communicates with the EVs registered by the EVD\@.
    As explained in the domain assumption section, we suppose that there is a third-party system that offers its API
    so to get the status of the battery of the EV. %An example of a system that provides these features is Smartcar,
    %which is already used by companies like AmpUp or BeCharge to remotely retrieve the battery level and remaining range of the EVs.
    \item \textbf{.}
\end{itemize}


\section{Functional Requirements}
\label{sec:functional_requirements}%


\section{Performance Requirements}
\label{sec:performance_requirements}%


\section{Design Constraints}
\label{sec:design_constraints}%

\subsection{Standards compliance}
\label{subsec:standards_compliance}%

\subsection{Hardware limitations}
\label{subsec:hardware_limitations}%

\subsection{Any other constraint}
\label{subsec:any_other_constraint}%


\section{Software System Attributes}
\label{sec:software_system_attributes}%

\subsection{Reliability}
\label{subsec:reliability}%

\subsection{Availability}
\label{subsec:availability}%

\subsection{Security}
\label{subsec:security}%

\subsection{Maintainability}
\label{subsec:maintainability}%

\subsection{Portability}
\label{subsec:portability}%
